% --- Introdução --- % (exemplo de capítulo sem numeração, mas presente no Sumário)

\chapter[Introdução]{Introdução}
% \addcontentsline{toc}{chapter}{Introdução}

% ----------------------------------------------------------

Na realidade do ensino superior brasileiro, mas não limitado a ele, nem todos estudantes são aprovados em todas as disciplinas que escolheram cursar em determinado semestre. Assim, causando um atrito na possível simetria na demanda dos alunos por matérias ao longo dos semestres. Além desta questão, surge também a decisão dos coordenadores de curso quanto a qual professor será associado a qual disciplina, bem como, quais serão essas disciplinas e em qual sala devem ser alocadas.

Dada grande quantidade de variáveis interconectadas e as características específicas de cada instituição \cite{MIRANDA2012505}, a organização destas informações buscando a melhor solução possível apresenta-se como um desafio. Principalmente se considerarmos que esta solução é buscada manualmente, estando também passível de erros humanos.

Embora seja um problema atual, isso não significa que seja recente. Desde 1978 \cite{BARHAM1978} o termo \textit{timetabling} encontra-se no meio acadêmico como o termo referente ao tabelamento de grade horária, sendo assim, é este o termo que será principalmente utilizado neste trabalho. Neste artigo de 1978 já se propunha uma forma para que se obtivesse um tabelamento otimizado, e demonstrava que o método utilizado gerava bons resultados.